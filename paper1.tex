%\documentclass[varwidth]{standalone}
\documentclass[10pt]{amsart}
\usepackage{amscd,amsxtra,color,amsthm}

\usepackage[all]{xy}
\usepackage{etex}
\usepackage{pictex}
\usepackage{graphicx}
\usepackage{tikz}
\usepackage[utf8]{inputenc} 
\usepackage[T1]{fontenc}
\usepackage[all]{xy}
\usepackage{etex}
\usepackage{pictex}
\usepackage{graphicx}
\usepackage{mathtools}
\DeclarePairedDelimiter{\ceil}{\lceil}{\rceil}
\DeclarePairedDelimiter{\floor}{\lfloor}{\rfloor}
\usepackage{comment}


\textheight=9in \textwidth=6.2in \topmargin=0in
\oddsidemargin=.15in \evensidemargin=.15in

\begin{document}
\parskip10pt
\parindent12pt
\baselineskip16pt





%%%%%%%%%%%%%%%%%%%%%%%%%%%%%%%%%%%%%%%%%%%%%%%%%%%%%%%%%%%%%%%%%%%%%%%%%%%%%%%%%%%%%%%%%%%%%%%%
%%  Definitions
%%%%%%%%%%%%%%%%%%%%%%%%%%%%%%%%%%%%%%%%%%%%%%%%%%%%%%%%%%%%%%%%%%%%%%%%%%%%%%%%%%%%%%%%%%%%%%%%

\def\G{\widetilde{G}}
\def\B{\widetilde{B}}
\def\T{\widetilde{T}}
\def\C{\mathbb{C}}
\def\A{\mathbb{A}}
\def\Z{\mathbb{Z}}
\def\R{\mathbb{R}}
\def\Q{\mathbb{Q}}
\def\N{\mathbb{N}}
\def\C{\mathbb{C}}
\def\F{\mathbb{F}}
\def\I{\mathbb{I}}
\def\H{\mathcal{H}}
\def\e{\varepsilon}
\def\s{\underline s}
\def\z{\zeta }
\def\vp{\varpi }
\def\O{\mathcal O}
\def\v{\upsilon }
\def\U{\Upsilon }
\def\p{\wp }
\def\p{\mathfrak{p}}
\def\B{\mathfrak{B}}

\newtheorem{theorem}{Theorem}%[section]
\newtheorem{lemma}[theorem]{Lemma}


\title{Proof for Closed Eulerian Trails}

\author{Graham Swain}
%\centerline{Your name goes here.}




\begin{abstract}
In this paper we explore a proof showing that a graph contains a closed Eulerian trail if and only if
all of its vertices have even degree. We show part of this proof by developing an algorithm.
\end{abstract}

\maketitle



%\begin{abstract}
%abstract goes here....
%\end{abstract}

%\maketitle

\section{Introduction}

This problem dates back to 18th century K\"{o}nigsberg, Prussia, modern day Kaliningrad, Russia. K\"{o}nigsberg had seven
bridges that connected four landmasses, the figure below shows roughly how the bridges were arranged. The citizens of
K\"{o}nigsberg wonder if a person could walk across town and cross each bridge exactly once.

\begin{figure}[h!]
\centerline{
{\includegraphics[width=.7\textwidth]{pictures/bridge_pic.png}}}\label{bridge}
\end{figure} 

\section{Theorem Environments}

\begin{theorem}[Ramsey, 1930] \ This is how you create a theorem. The reference next to the Theorem name can be left out.  \label{theorem1}
\end{theorem}

\begin{proof} \ This is how you can create a proof environment.  \end{proof}

In the tex file, note that a label was added within the above theorem environment.  This is so that you can refer to Theorem \ref{theorem1} and if you add more theorems to the paper, they will automatically be renumbered.  The same is true for references.  You may refer to \cite{R}.  It may be necessary to compile the tex file twice before the labels show up.  All sources used in your paper should be listed as in the samples below.  The first, second, and fifth references are articles, while the third and fourth are books.

Here is a sample of a math environment: $\sqrt[4]{14}$.  Use double $\$$ if you wish to have a math environment centered on a line by itself:
$$\mathop{\int}\limits_{1}^{\infty} e^{-x} \ dx.$$
If you would like to have an equation be numbered, use the following:
\begin{equation}  4x^5+3y^7=85. \label{eq} \end{equation}
You can then refer label in the equation.  Eg., equation (\ref{eq}) pulls up the correct number.

Equations, inequalities, etc... can be aligned using the following commands:
\begin{align} \delta (H) &= n-(r-1)-\Delta (\overline{H}) \notag \\
                                    &\ge n-(r-1)-(m+2) \notag \\
                                    &\ge n-m-r-1 \label{ineq} \end{align}
As with the equation environment, each line that has $\backslash$notag will not be labelled, and putting a label allows you to refer to property (\ref{ineq}).

Matrices can be created using the array command:

$$\left( \begin{array}{rrr} -2 & 3 & -7 \\ 2 & 0 & 4 \end{array} \right)$$
% Note that the 3 r's after the above array command right-align the entries.  Replacing any of them with l or c will left-align or center the corresponding columns.

\bibliographystyle{amsplain}
\begin{thebibliography}{10}

\bibitem{C} V. Chv\'atal, {\it Tree-complete Graph Ramsey Numbers,}  J. Graph Theory {\bf 1} (1977), 93.

\bibitem{CH} V. Chv\'atal and F. Harary, {\it Generalized Ramsey Theory for Graphs III. Small Off-diagonal Numbers,} Pacific J. Math. {\bf 41} (1972), 335-345.

\bibitem{IR} K. Ireland and M. Rosen, ``A Classical
Introduction to Modern Number Theory,'' $2^{nd}$ edition,
Springer-Verlag, 1990.

\bibitem{J} G. Janusz, ``Algebraic Number Fields,'' $2^{nd}$ edition, Graduate Studies in Mathematics {\bf 7}, American Mathematical Society, Providence, RI, 1996.

\bibitem{R} F. Ramsey, {\it On a Problem of Formal Logic,} Proc. London Math. Soc. {\bf 30} (1930), 264-286.



\end{thebibliography}


\end{document}
